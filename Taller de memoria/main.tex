\documentclass{article}
\usepackage[utf8]{inputenc}
\usepackage[spanish]{babel}
\usepackage{listings}
\usepackage{graphicx}
\graphicspath{ {images/} }
\usepackage{cite}

\begin{document}

\begin{titlepage}
    \begin{center}
        \vspace*{1cm}
            
        \Huge
        \textbf{Taller}
            
        \vspace{0.5cm}
        \LARGE
        Nociones de la memoria del computador
            
        \vspace{1.5cm}
            
        \textbf{Laura Maria Zuluaga Jaramillo}
        
        \vspace{0.8cm}
        
        \begin{figure}[h]
        \includegraphics[width=6cm]{udea.png}
        \centering
        \label{fig:udea}
        \end{figure}
            
        \vfill
            
        \vspace{0.8cm}
         
        \Large
        Despartamento de Ingeniería Electrónica y Telecomunicaciones\\
        Universidad de Antioquia\\
        Medellín\\
        Septiembre de 2020
            
    \end{center}
\end{titlepage}

\tableofcontents

\section{Introducción}
Es muy común pensar que las memorias electrónicas se encuentrar solo en los computadores, pero actualmente, la gran mayoria de dispositivos electrónicos cuentan con una de ellas para su funcionamiento. Por ende, es importante entender el concepto de memoria, conocer su importancia, funcionamiento y que tipos existen. Esto se logrará por medio de cuatro interrogantes que seran respondidas a continuación.

\section{Interrogantes resultas en cuanto a la memoria} \label{contenido}

\begin{enumerate}
    \item Defina que es la memoria del computador.\\
    La memoria, es un componente imprescindible en los computadores, la cual de una manera general se puede describir como la encargada de almacenar o retener temporalmente la información y/o programas manteniendola disponible para que posteriormente los microprocesadores la ejecuten, modifiquen y la regresen a la memoria para mostrar los resultados requeridos por el usuario. \cite{ecured}
    
    De una manera mas detallada podemos decir que la información que se toma del disco duro es llevada a la memoria para trabajarla, luego es entregada al procesador para que realice las operaciones necesarias y realizar modificaciones en estos datos, la información ya modificada es devuelta a otra sección de la memoria para posteriormente ser guardada en el disco duro.
    El microprocesador utiliza la memoria para trabajar con información de gran tamaño procesandola por partes ya que este no cuenta con suficiente capacidad de almacenamiento.
    Tecnicamente se llama memoria a todo dispositivo de almacenamiento pero usualmente se utiliza el termino para referirse a dispositivos de almacenamiento temporal y alta \textbf{velocidad de acceso}. \cite{augusto}
    
\end{enumerate}

En la sección de teoremas (\ref{contenido})

\section{Conclusión} \label{conclulsion}

\bibliographystyle{IEEEtran}
\bibliography{references}

\end{document}
